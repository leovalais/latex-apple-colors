\documentclass[a4paper]{article}
\usepackage[utf8]{inputenc}
\usepackage[T1]{fontenc}
\usepackage{pgffor}
\usepackage[table]{xcolor}
\usepackage{longtable}
\usepackage{applecolors}

\title{Demo of \texttt{applecolors}}
\author{Léo Valais}
\date\today

\newcommand{\col}[1]{\textcolor{#1}{\texttt{#1}}}
\newcommand{\colb}[1]{\colorbox{black}{\col{#1}}}
\newcommand{\cbox}[2][\qquad]{\colorbox{#2}{#1}}

\begin{document}
\maketitle

\newcommand{\showc}[1]{%
    \begin{itemize}%
    \foreach \c in {#1}{%
    \item \col{\c}\qquad\colb{\c}\qquad\cbox{\c}
    }%
    \end{itemize}}


\section{iOS colors}
\showc{iOS-pink,iOS-red,iOS-orange,iOS-yellow,iOS-green,iOS-tealblue,iOS-blue,iOS-purple}

\section{watchOS colors}
\showc{watchOS-pink,watchOS-red,watchOS-orange,watchOS-yellow,watchOS-green,watchOS-turquoise,watchOS-cyan,watchOS-blue,watchOS-purple,watchOS-white}

\section{macOS colors}
Colors \texttt{macOS-acc-\textcolor{red}{*}} are designed for accessibility.

\bigskip

\newcommand{\putrow}[3][white]{%
    \cellcolor{#1}\col{macOS#3-#2} &%
    \cellcolor{#1}\col{macOS#3-aqua#2} &%
    \cellcolor{macOS#3-#2} &%
    \cellcolor{macOS#3-aqua#2} &%
    \cellcolor{black}\col{macOS#3-#2} &%
    \cellcolor{black}\col{macOS#3-aqua#2}}

\hspace*{-4cm}
\footnotesize
\begin{tabular}{cccccc}
    \hline
    Normal & Aqua & Normal BG & Aqua BG & Black BG & Black BG \\
    \hline
    \putrow{blue}{}\\
    \putrow{brown}{}\\
    \putrow{gray}{}\\
    \putrow{green}{}\\
    \putrow{orange}{}\\
    \putrow{pink}{}\\
    \putrow{red}{}\\
    \putrow{yellow}{}\\
    \hline
    \putrow{blue}{-vibrant}\\
    \putrow{brown}{-vibrant}\\
    \putrow{gray}{-vibrant}\\
    \putrow{green}{-vibrant}\\
    \putrow{orange}{-vibrant}\\
    \putrow{pink}{-vibrant}\\
    \putrow{red}{-vibrant}\\
    \putrow{yellow}{-vibrant}\\
    \hline
    \putrow{blue}{-acc}\\
    \putrow{brown}{-acc}\\
    \putrow{gray}{-acc}\\
    \putrow{green}{-acc}\\
    \putrow{orange}{-acc}\\
    \putrow{pink}{-acc}\\
    \putrow{red}{-acc}\\
    \putrow{yellow}{-acc}\\
    \hline
    \putrow{blue}{-acc-vibrant}\\
    \putrow{brown}{-acc-vibrant}\\
    \putrow{gray}{-acc-vibrant}\\
    \putrow{green}{-acc-vibrant}\\
    \putrow{orange}{-acc-vibrant}\\
    \putrow{pink}{-acc-vibrant}\\
    \putrow{red}{-acc-vibrant}\\
    \putrow{yellow}{-acc-vibrant}\\
    \hline
\end{tabular}
\end{document}

